\documentclass[a4paper,10pt]{article}
\usepackage[utf8]{inputenc}
\usepackage{url}
%\usepackage{cleveref}

\renewcommand{\abstractname}{Resumen}
\renewcommand{\refname}{Bibliografía}
%opening
%\title{Producción de Biodiesel por Cavitación Hidrodinamica apartir de Aceite usado de cocina}
%\title{Cinetica de Transesterificaci\'on del Biodiesel por Sonicaci\'on apartir de Aceite Usado de Cocina}
\title{Optimizaci\'on del tiempo de Transesterificaci\'on del Biodiesel por Sonicaci\'on apartir de Aceite Usado de Cocina}
\author{Juan Miguel Ravelo Jove}
\date{27 Octubre, 2017}
\begin{document}

\maketitle

\begin{abstract}
El agotamiento de los combustibles fosiles, problemas medioambientales, y el encarecimiento del precio del crudo ha incentivado 
la busqueda de combustibles alternativos. Las caracteristicas del biodiesel ha hecho apuntar a una producción atractiva de biodiesel 
de alta calidad. El uso de aceite usado de cocina es el componente clave para la reducción del coste de producción a un $60-90\%$. Este
estudio se enfoca en la producción de biodiesel apartir de aceite usado de cocina, bajo las \emph{implosiones} de 
la \emph{cavitación} usando hidroxido de potasio ($KOH$)  como catalizador.
\end{abstract}

\paragraph{Palabras clave:} 
\begin{itemize}
 \item Aceite usado de cocina.
 \item Biodiesel.
 \item Cavitación.
 \item Energia de activación.
 \item Transesterificación.
\end{itemize}

\paragraph{Objetivo principal:}
%Minimizar los costos de producción de Biodiesel.   %%%goomy
Proponer un m\'odelo matem\'atico escalable para una planta piloto.
\paragraph{Objetivo secundario:}
Evaluar los efectos de la sonicación sobre la cinetica de producción en los siguientes topicos.
\begin{itemize}
 \item Energia de activaci\'on.
 \item Tiempo de reacción.
 \item Conversión del ester metilico.
\end{itemize}


\section{Introducción}
Existe varias razones para la busqueda de un combustible alternativo, factibilidad tecnica,
 ambientalmente aceptable, economicamente competitivo y facilmente disponible.
\begin{itemize}
 \item[1.a] La primer principal razón  es el incremento de la demanda de combustible fosiles en todos los sectores de la vida humana~\cite{Kafuku2010}.
    \begin{itemize}
	  \item Transporte en general.
	  \item Generación de energia.
	  \item Procesos industriales.
	  \item Consumo domestico.
    \end{itemize}
 \item[1.b] Esta creciente demanda da lugar a problemas ambientales, tales como grandes 
  emisiones de $CO_2$ y gases de efecto invernadero y tambien el calentamiento global.
 \item[1.c] El consumo mundial de energia se doblo entre 1970 y el 2001~\cite{talebian2013}.
 \item[1.d] El 2006 la \emph{International Energy Outlook} predijo que la demanda mundial de petroleo 
  se incrementara de 84.4 a 116 millones de barriles por dia (mb/d) en USA para el 2030.
 \item[1.e] Actualmente USA consume 96.63 mb/d según la \emph{International Energy Agency}~\cite{iea2017}.
	%https://www.iea.org/oilmarketreport/omrpublic/
 \item[2] La segunda razón es que las fuentes de combustibles fosiles no son renovables~\cite{maceiras2011}. % referencia 4
 \item[3] La ultima razón es la inestabilidad del precio del crudo~\cite{santori2012}. %ref 6
\end{itemize}

Según la \emph{OEFA} el Perú importa 205 mil barriles por dia y produce 63 mil barriles; simplificando el Perú de 3 barriles que
consume a diario, 2 los importa y 1 lo produce.
La capacidad del biodiesel de ser usado en lugar del petroleo es una de sus mas importantes caracteristicas~\cite{geyer1984} \\
Mezclas de 5(BD5) a 20\%(BD20) de biodiesel pueden ser usados en motores diesel existentes sin modificación~\cite{Ghorbani2011}.\\
%El biodiesel es un \emph{ester metilico} producido de aceite vegetal o de grasa animal $[8,9]$.
La ASTM (\emph{American Society for Testing and Materials}) define al biodiesel como un ester monoalquilico de cadena larga de acidos 
grasos, derivados de una fuente renovable lipidica, como aceite vegetal o grasa animal.\\
El mas importante obstaculo en la industrialización y comercialización del biodiesel es su costo de producción~\cite{talebian2013,demirbas2009}. 
Por lo tanto el uso del aceite usado de cocina, reduce su costo de producción en un $60-90\%$~\cite{talebian2013,canakci1998}. \\
Biodiesel tiene una influencia significativa en la reducción de emisiones del motor tales como hidrocarburos no 
quemados ($68\%$), particulas ($40\%$), monoxido de carbono ($44\%$), oxido de azufre ($100\%$) e 
hidrocarburos aromaticos policiclicos ($80-90\%$)~\cite{talebian2013,leduc2009}.

\section{Mecanismos de producción de biodiesel}

\subsection{Mecanismo de cavitación}
La cavitación es el proceso de nucleación en un liquido cuando la presión cae por debajo de la presión de vapor~\cite{khosravi2016,brennen2013}.
%El sistema es compuesto por un tanque reservorio conectado a una bomba centrifuga y a un motor de $1.5\:kW$. La tuberia de descarga fue conectada al tubo principal y  bypass. El tubo principal fue conectado directamente a un plato de orificios para generar la cavitación. Valvulas de estrangulamiento y medidores de presion fueron ajustados para medir la presion (Cravotto capitulo 3.,2015). En sintesis se espera mediante un estrangulamiento hacer caer la suficiente presión, para generar la nucleación del flujo (metanol/aceite), de modo tal que la implosion que esta genere, en la camara de cavitación, entregue la \emph{energia de activación} necesaria para su reacción.
Las ondas ultrasónicas de alta intensidad generan cavitación en líquidos. 
Las cavitaciones provocan efectos locales extremos, como chorros de líquido de hasta 1000 Km/h, 
presiones de hasta 2000 atm y temperaturas de hasta 5000 K~\cite{gude2013}. \\
Al sonicar líquidos a altas intensidades, las ondas de sonido que se propagan en el medio líquido, dan como resultado ciclos alternos 
de alta presión (compresi\'on) y baja presión (rarefacci\'on), con velocidades que dependen de la frecuencia. Durante el ciclo de baja presión, 
las ondas ultrasónicas de alta intensidad crean pequeñas burbujas  o vacío en el líquido,
Cuando las burbujas alcanzan un volumen en el que ya no pueden absorber energía, colapsan violentamente durante un ciclo de alta presión~\cite{khosravi2016,suslick1999}.

\subsection{Mecanismo de transesterificación}
La transesterificación de aceite vegetal con alcohol es el mejor método para la producción de biodiesel. La utilización de diferentes 
tipos de catalizadores mejora la cinetica y conversion del biodiesel. La transesterificación es una reacción reversible y el exceso de 
alcohol desplaza el equilibrio a los productos~\cite{talebian2013}. \\ %[56.57]
Las mas importantes variables son:
\begin{itemize}
 \item Temperatura de reacción.
 \item Contenido de acidos grasos libres en el aceite.
 \item Contenido de agua en el aceite.
 \item Tipo y cantidad de catalizador.
 \item Tiempo de reacción.
 \item Proporción metanol:aceite.
 \item Intensidad de mezcla.
\end{itemize}

\section{Materiales y métodos}
El aceite usado de cocina es provisto por restaurantes y pollerias de la localidad de Arequipa. 
El metanol ($CH_3 OH$) al ($99\%$) y el catalizador hidroxido de potasio ($KOH$) se proveran por Peruquimicos S.A.C. 
La conversíon del aceite en ester metílico por cromatografia de gases (GC).
\subsection{Pretratamiento del aceite usado de cocina}
La reacción alquílica es muy sensible al contenido de agua y acidos grasos libres~\cite{khosravi2016,zhang2003}.% $[96,97]$. 
Una primera filtración eliminará restos solidos, ya filtrado, queda eliminar el agua en un decantador y finalmente nos queda 
los (acidos grasos libres) FFA, el cual se reducira en un filtro de zeolita.
\subsection{Procedimiento experimental}
Se hace una pre-reacción entre el $CH_3 OH$ y el $KOH$ para obtener el metoxido, elemento cual cataliza la reacción de transesterificación, 
reduciendo el tiempo de reacción.
Con esto el paso de la mezcla metanol:aceite:catalizador por la camara de cavitación, dado que las implosiones suministraran energia 
suficiente para completar la transesterificación.
La glicerina obtenia es separada por decantación. El biodiesel obtenido una vez enfriado es filtrado en azerrin, 
lo cual atrapa las moleculas restantes de glicerina. 
\paragraph{Material pretratamiento al aceite}
\begin{itemize}
 \item Cilindro de 15 y 60 gln.
 \item Criba.
 \item Malla 150.
 \item Tanque decantador de 100 gln.
 \item Baterial de filtros de 25 gln.
 \item arena.
 \item zeolita.
\end{itemize}

\paragraph{Material de Reacción-equipos}
\begin{itemize}
 \item Tanque Reactor-Mezclador de 50 gln
 \item Tanque $CH_3 OH-KOH$
 \item Intercambiador de calor
 \item Bomba centrifuga
 \item Bomba hidraulica
 \item Camara de Cavitación
 \item Tanque deposito, decantador (100 gln)
\end{itemize}

\paragraph{Zona de filtrado del Biodiesel}
\begin{itemize}
 \item Tanque deposito, biodiesel sucio
 \item Bomba hidraulica
 \item Tanque filtro de azerrin
 \item Bateria de filtro en $\mu$m
 \item Tanque deposito biodiesel limpio
\end{itemize}

\subsection{Cromatografia de gases GC}
El  analisis del biodiesel se realizará por cromatografia de gases con TR-CN100 columna \emph{Teknokroma} 
a $185^o C$ de temperatura y un detector FID a $260^o C$. ASTM D1983  es el estandar para este analisis.

\bibliographystyle{unsrt}
\bibliography{biblio}
\end{document}
